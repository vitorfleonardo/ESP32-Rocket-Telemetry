\chapter[Introdução]{Introdução}

A Universidade de Brasília (UnB), por meio da Faculdade de Ciências e Tecnologias em Engenharia (FCTE), promove uma formação interdisciplinar com foco em inovação tecnológica e desenvolvimento aplicado, integrando as engenharias Aeroespacial, Automotiva, de Energia, de Software e Eletrônica. No contexto do curso de Engenharia, a disciplina de Projeto Integrador 1 propõe a realização de projetos colaborativos, desafiando os estudantes a aplicarem conhecimentos técnicos, gerenciais e científicos na resolução de problemas reais. A ementa da disciplina compreende noções de projeto e gestão, modelos de ciclo de vida, gerenciamento de escopo, tempo, qualidade, recursos humanos e riscos, além da prática por meio de projetos interdisciplinares.

Entre as propostas de desafio prático, destaca-se o desenvolvimento de um sistema de lançamento de foguetes d’água com controle de trajetória. Essa atividade visa estimular competências de engenharia aplicada, prototipagem, controle e automação, além de considerar aspectos de segurança e reusabilidade. O objetivo do projeto é construir um foguete com propulsão hidrostática capaz de atingir distâncias predefinidas de 10m, 20m e 30m, com precisão igual ou inferior a ±0,5 m, utilizando uma base de lançamento automatizada. A solução deve assegurar distanciamento mínimo de 5 metros de pessoas envolvidas e reaproveitamento do foguete em três ciclos de lançamento.

A utilização de propulsão à base de água (ou propulsão hidrostática) como método educacional tem sido recorrente em instituições de ensino devido ao seu baixo custo, segurança e potencial para experimentação científica. Diversos trabalhos acadêmicos relatam o uso de foguetes PET e sistemas automatizados em ambientes escolares e universitários. \cite{yukimitsu2020}, do Instituto Federal de Campinas, desenvolveram uma base lançadora automatizada para participação em olimpíadas científicas, empregando sensores e microcontroladores para controle de pressão e segurança do disparo, demonstrando a viabilidade de sistemas eletromecânicos embarcados nesse tipo de aplicação. De forma semelhante, Guimarães e Francisco \cite{guimaraes2020foguete} propuseram uma plataforma automatizada com redução de torque para lançamento de foguetes PET, visando maior estabilidade e controle sobre o ângulo de disparo, destacando a importância da calibração e automação nos mecanismos de acionamento.

Do ponto de vista regulatório, embora foguetes experimentais de pequeno porte não sejam classificados como artefatos sujeitos à regulação pela Agência Espacial Brasileira (AEB), normas de segurança devem ser observadas. Segundo a National Association of Rocketry  \cite{nar2012safety}, recomenda-se uma distância mínima de cinco metros para lançamentos de foguetes de baixa potência. Tais diretrizes são utilizadas como base para a definição dos parâmetros de segurança neste projeto.

O desenvolvimento de soluções próprias, em detrimento do uso de sistemas prontos, é estimulado no escopo da disciplina Projeto Integrador I, promovendo originalidade, inovação e domínio tecnológico. Neste contexto, a integração entre sensores de pressão, giroscópios, acelerômetros e módulos de comunicação sem fio, todos conectados a um microcontrolador como o ESP32, permite realizar medições em tempo real, persistir dados para análises futuras e otimizar a trajetória do foguete por meio de ajustes iterativos.

Dessa forma, o presente projeto justifica-se pela sua contribuição ao ensino prático de engenharia, pelo incentivo à construção de soluções seguras, reutilizáveis e tecnicamente viáveis, e pela proposta de um projeto que combina as engenharias de Software, Automotiva, Aeroespacial, Eletrônica e Energia. Além disso, atende a uma demanda acadêmica por experiências educacionais multidisciplinares que promovam competências técnicas e habilidades de trabalho em equipe.

Dessa forma, o projeto justifica-se pela necessidade de um sistema didático acessível para o ensino prático de engenharia, que integre eletrônica, software, estruturas e energia em uma solução reutilizável e segura. A proposta substitui sistemas prontos por uma base de lançamento desenvolvida do zero, com controle automatizado e coleta de dados em tempo real, permitindo experimentação e análise. Ademais, atende à demanda acadêmica por projetos interdisciplinares, promovendo competências técnicas e habilidades de trabalho em equipe.
