\chapter{Análise de mercado}

\section{Nome comercial e conceito para o produto}

\textcolor{red}{Especifique o nome do projeto, e use uma ou duas frases curtas e objetivas para descrever o conceito do produto.}

\section{Mercado-alvo}

\subsection{Consumidor final}

\textcolor{red}{Pessoas, empresas, instituições etc. que usufruirão dos produtos, serviços e resultados gerados pelo projeto, cujos requisitos (tópico abaixo) devem atender as suas necessidades. Podem ser internas ou externas à organização, mas, merecem destaque especial, pois, o projeto está sendo feito para atendê-los de forma direta ou indireta.}

\subsection{Cliente responsável pela comercialização}

\textcolor{red}{Se a comercialização do produto desenvolvido necessitará de um distribuidor e/ou um tipo específico de empresa, como lojas de atacada ou varejo em geral, descreva aqui.}

\section{Requisitos}

\subsection{Consumidor final}

\textcolor{red}{Listar os fatos que são essenciais para o consumidor adquirir o produto, exemplo: cor, tamanho, material, tempo de vida-útil, dispositivo de segurança \textbf{x}, entre outros.}

\subsection{Cliente responsável pela comercialização}

\textcolor{red}{Principais requisitos que o cliente responsável pela distribuição/venda do produto necessita.}

\subsection{Legislação/Normas}

\textcolor{red}{Listar os fatos legais que devem ser atendidos pelo produto e não podem ser alterados, limitando as opções da equipe do projeto.}

\section{Justificativa}

\textcolor{red}{Informar o problema ou a oportunidade (necessidade) que justifica o porquê de o projeto ser realizado. Por exemplo: atende uma demanda específica do consumidor final; supre uma necessidade do mercado comercializador; é um diferencial X para o órgão regulamentador.}

\section{Indicadores}

\textcolor{red}{Listar até 10 indicadores que determinam o mercado consumidor do produto desenvolvido: exemplo: 1) n° de alunos da FGA que utilizam ônibus às 18:00; 2) n° de usuários do restaurante universitários, 3) número de idosos classificados como público-alvo no DF e no estado de Goiás, 4) n° de empresas de segurança registradas no DF etc.}

\section{Concorrência}

\subsection{Nome do concorrente 1}

\textcolor{red}{Listar nome do produto, preço em reais, principais características
positivas e principais características negativas.}

\subsection{Nome do concorrente 2}

\textcolor{red}{Listar nome do produto, preço em reais, principais características
positivas e principais características negativas.}

\subsection{Nome do concorrente N}

\textcolor{red}{Listar nome do produto, preço em reais, principais características
positivas e principais características negativas.}
