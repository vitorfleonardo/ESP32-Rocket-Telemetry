
% \section{Experimentos de consumo energético}

% \textcolor{red}{Apresentar com detalhes os experimentos feitos para conferir se as demandas energéticas do projeto foram atendidas.}

\section{Análise de Consumo Energético}

O dimensionamento energético apresenta-se como um dos pilares fundamentais no Projeto Integrador de Engenharia 1, sendo determinante para o êxito do sistema, desde o lançamento do foguete até o fornecimento contínuo de energia aos componentes eletrônicos responsáveis pela automação completa da solução. 

O consumo energético de um projeto depende dos componentes utilizados, da forma como são alimentados e da eficiência do gerenciamento de energia. Cada elemento no sistema, como sensores, microcontroladores, motores e módulos de comunicação possuem um consumo específico, expresso em corrente e/ou tensão. 

A busca pela eficiência da gestão energética visa equilibrar o desempenho e a autonomia do sistema embarcado do foguete evitando desperdícios e garantindo que o sistema opere dentro das limitações da fonte de energia disponível. 

% Identificação dos Subsistemas Elétricos e Especificações dos Componentes 
\subsection{Subsistemas Elétricos e Componentes}

A eficiência energética de um foguete automatizado é um fator determinante para seu desempenho e autonomia operacional. Para compreender o consumo de energia do sistema, medições, análises teóricas e experimentais com o intuito de caracterizar o comportamento do fluxo energético do foguete em cenários de repouso e movimentação contínua foram realizados.  

Para uma análise detalhada do consumo energético, o projeto foi dividido em dois subsistemas eletrônicos principais: a base de lançamento e o foguete. Essa separação permite uma caracterização precisa do consumo em cada fase dos testes.

Os dispositivos eletroeletrônicos empregados no projeto, juntamente com suas especificações de tensão e corrente média, são detalhados na tabela a seguir: 

\begin{table}[H]
    \centering
    \caption{Informações elétricas dos componentes}
    \label{tab:componentes_todos}
    \begin{tabular}{|l|l|l|l|}
        \hline
        Nome & Unidades & Tensão (V) & Corrente (mA) \\
        \hline
        ESP-32 & 2 & 3.3 & 160 \\
		\hline
        Motor DC & 1 & 3.3 & 1000 \\
		\hline
		\makecell[l]{Módulo Matriz \\ LED 8×8 (MAX7219)} & 1 & 3.3 & 160 \\
		\hline
		\makecell[l]{Módulo Ponte H \\ Dupla L298N} & 1 & 3.3 & 20 (standby) 1000 (ativos) \\
		\hline
		\makecell[l]{Módulo Interruptor \\ de 3 pinos} & 1 & 3.3 & 1 \\
        \hline
		\makecell[l]{Módulo de Cartão SD} & 1 & 3.3 & 100 \\
		\hline
		\makecell[l]{Módulo MPU-6050} & 1 & 3.3 & 3.9 \\
		\hline
    \end{tabular}
\end{table}

\subsubsection{Base de Lançamento}

Este subsistema engloba os componentes eletrônicos responsáveis pelo controle e monitoramento da fase de pré-lançamento e do processo de pressurização do foguete. Seu funcionamento é crítico para o sucesso da decolagem. Os componentes que consomem energia dentro desse subsistema e suas funções são: 

\begin{table}[H]
    \centering
    \caption{Informações elétricas dos componentes da base de lançamento}
    \label{tab:componentes_base}
    \begin{tabular}{|l|l|}
        \hline
        Nome & Função \\
        \hline
        ESP-32 & \makecell[l]{Atua como o microcontrolador \\ central da base, gerenciando a comunicação, \\ os sensores da plataforma \\ e os acionadores.} \\
		\hline
        Motor DC & \makecell[l]{Utilizado para fins de \\ acionamento eletromecânico do mecanismo \\ de lançamento do foguete.} \\
		\hline
		Módulo Matriz LED 8x8 & \makecell[l]{Empregado para fornecer feedback \\ visual para o countdown do \\ lançamento do foguete.} \\
		\hline
    \end{tabular}
\end{table}

\subsubsection{Subsistema do Foguete}

Este subsistema engloba toda a eletrônica embarcada na coifa do foguete, vital para a coleta dos dados durante o lançamento. O funcionamento é essencial para o sucesso do projeto. Os componentes que consomem energia no subsistema e suas funções são: 

\begin{table}[H]
    \centering
    \caption{Informações elétricas dos componentes do foguete}
    \label{tab:componentes_foguete}
    \begin{tabular}{|l|l|}
        \hline
        Nome & Função \\
        \hline
        ESP-32 & \makecell[l]{Atua como o cérebro embarcado do foguete, \\ processando dados dos sensores, armazenando \\ informações e controlando sistemas internos.} \\
		\hline
        Módulo de Cartão SD & \makecell[l]{Utilizado para armazenamento de dados do voo, \\ apresentando consumo elevado de energia durante \\ operações de escrita.} \\
		\hline
		Módulo MPU-6050 & \makecell[l]{Sensor inercial que fornece dados de aceleração \\ e rotação, sendo operado continuamente durante \\ o voo.} \\
		\hline
		Módulo Interruptor de 3 pinos & \makecell[l]{Utilizado para reiniciar o sistema de leitura do \\ microcontrolador ESP32.} \\
		\hline
    \end{tabular}
\end{table}

\subsection{Estimativa de Duração do Lançamento}

Para o cálculo da energia consumida (J) de cada item, o tempo estimado de Uso (s) deve ser conhecido para evitar desperdícios de energia. Com isso, consideramos um cenário de lançamento que abrange as fases de pré-lançamento, voo e recuperação. O tempo de pré-lançamento e recuperação são considerados constantes para todos os cenários, representando o tempo necessário para inicializar o sistema e recuperar os dados. Já o tempo de voo será proporcional à distância que o foguete deve percorrer. Assim, estabelecemos três configurações padrão: 
 
\begin{table}[H]
	\centering
	\caption{Tempos estimados por configuração de lançamento}
	\label{tab:duracao_lancamento}
	\begin{tabular}{|l|c|c|c|}
		\hline
		Configuração & Pré-lançamento (s) & Voo (s) & Recuperação (s) \\
		\hline
		10 metros & 120 & 3 & 180 \\
		\hline
		20 metros & 120 & 3 & 180 \\
		\hline
		30 metros & 120 & 3 & 180 \\
		\hline
	\end{tabular}
\end{table}

As estimativas de tempo são baseadas em modelos de voo simplificados para foguetes de pequena escala, discutidas entre os membros da equipe, com alunos que já cursaram a disciplina Projeto Integrador de Engenharia 1 e tiveram a oportunidade de construir o projeto voltado ao foguete e fontes externas de pesquisa sobre lançamentos de foguetes manuais, prática bastante difundida nos ensinos fundamental e médio em todo Brasil. 

\subsubsection{Cálculo da Energia Consumida}

A energia consumida por cada componente é calculada utilizando a fórmula:

\begin{equation}
E = V \cdot I \cdot t
\end{equation}

onde $E$ é a energia em Joules (J), $V$ é a tensão em Volts (V), $I$ é a corrente em Ampères (A) e $t$ é o tempo em Segundos (s). As correntes em miliampères (mA) foram convertidas para Ampères (A) dividindo por 1000.

\subsubsection{Cálculos para o Subsistema da Base de lançamento}

Os componentes da base operam principalmente durante a fase de pré-lançamento. O Motor DC, embora acionado brevemente, possui uma corrente elevada durante sua ativação. O consumo para a base é considerado fixo para todos os cenários.

\begin{table}[H]
    \centering
    \caption{Tabela de consumo de energia dos componentes da base}
    \label{tab:consumo_componentes_base}
    \begin{tabular}{|l|c|c|c|c|c|}
        \hline
        Componente & Quantidade & Tensão (V) & Corrente (A) & Tempo (s) & Energia (J) \\
        \hline
        ESP32 & 1 & 3.3 & 0.160 & 120 & 63.36 \\
        \hline
        Motor DC & 1 & 3.3 & 1 & 5 & 16.50 \\
        \hline
        Matriz LED 8X8 & 1 & 3.3 & 0.160 & 120 & 63.36 \\
        \hline
        Ponte H L298N & 1 & 3.3 & 0.020 & 120 & 7.92 \\
        \hline
        \multicolumn{5}{|l|}{Total do Subsistema da Base} & 151,14 \\
        \hline
    \end{tabular}
\end{table}

\subsubsection{Cálculos para o Subsistema do Foguete}

Os componentes do foguete operam desde o pré-lançamento (enquanto o foguete está na base), durante o voo e na fase de recuperação. O tempo total de operação do foguete (pré-lançamento, mais voo, mais recuperação) varia ligeiramente conforme o cenário de voo. 

Para os cálculos abaixo, utilizaremos o cenário para 10 metros como exemplo, onde o tempo total de uso no foguete é de 307 segundos (120 s de pré-lançamento +7 s de voo +180 s de recuperação).

\begin{table}[H]
    \centering
    \caption{Tabela de consumo de energia dos componentes do foguete}
    \label{tab:consumo_componentes_foguete}
    \begin{tabular}{|l|c|c|c|c|c|}
        \hline
        Componente & Quantidade & Tensão (V) & Corrente (A) & Tempo (s) & Energia (J) \\
        \hline
        ESP32 & 1 & 3.3 & 0.160 & 307 & 162.26 \\
        \hline
        Cartão SD & 1 & 3.3 & 0.100 & 307 & 101.31 \\
        \hline
        MPU-6050 & 1 & 3.3 & 0.0039 & 307 & 3.96 \\
        \hline
        Interruptor 3 pinos & 1 & 3.3 & 0.001 & 307 & 1.01 \\
        \hline
        \multicolumn{5}{|l|}{Total do Subsistema do Foguete} & 268,54 \\
        \hline
    \end{tabular}
\end{table}

\subsubsection{Cálculo do Consumo Total de Energia}

Isso significa que a base de lançamento terá um consumo fixo de 151,14 J por lançamento e o foguete terá um consumo de aproximadamente 268,54 J por lançamento. Considerando o consumo total para as três configurações necessárias para o sucesso do projeto, o total de energia utilizado será 1.259,04 J, ou aproximadamente 0,350 Wh. 

\subsection{Dimensionamento e Seleção das Fontes de Alimentação}

Com os valores de energia em Wh definidos, partimos para a escolha das fontes de alimentação. Para o dimensionamento adequado, consideramos a eficiência típica de circuitos elevadores de tensão, que leva em conta as inevitáveis perdas na conversão e as variações de tensão durante a operação.

% \subsubsection{Fator de Eficiência do Módulo Elevador de Tensão}
% Para o módulo boost MT3608 da base de lançamento, adotamos uma eficiência de 85\%. Esse valor combina dados do fabricante (datasheet) e resultados de testes práticos, alinhando-se ao intervalo típico de operação (85–93\%) com entrada em 3,7 V e saída em 3,3 V. Assim, as perdas do conversor já estão previstas no cálculo da capacidade da bateria do foguete, garantindo cobertura realista das necessidades energéticas.

\subsubsection{Margem de Segurança}

A aplicação de uma margem de segurança de 30\% sobre o consumo energético total é uma prática de engenharia fundamental e amplamente aceita em projetos de sistemas embarcados e aeroespaciais, onde a falha de energia é crítica. Esta margem visa mitigar incertezas e garantir a robustez do sistema frente a condições imprevistas: variações e degradação da bateria descritas a seguir.

% Temperatura:
A capacidade de descarga de baterias de LiPo é significativamente afetada por temperaturas extremas. Em baixas temperaturas, a capacidade disponível pode ser reduzida em 20\% ou mais. Em altas temperaturas, a degradação a longo prazo e o envelhecimento da bateria são acelerados.\cite{plett2015}

% Ciclos de Carga e Descarga:
Baterias de íon-lítio perdem capacidade progressivamente com cada ciclo de carga ou descarga e com o tempo de uso (envelhecimento). 

% Picos de Corrente Inesperados e Variações de Consumo: 
As correntes médias calculadas são estimativas; picos de corrente transitórios (ex: inicialização de módulos, escrita no SD) podem exceder a média e são absorvidos pela margem. Variações de lote entre componentes e imprecisões nos datasheets também são compensadas. 

% Cenários Operacionais Imprevistos: 
A margem permite que o nosso sistema opere por um tempo e talvez maior em caso de atrasos na recuperação ou necessidade de retransmissão de dados, aumentando a probabilidade de sucesso do voo. 

% Conformidade com Boas Práticas de Engenharia:  
Em projetos críticos, margens de segurança para recursos como potência são padrão. \cite{nasa2016} Margens entre 20\% e 50\% são comuns para alta confiabilidade. 

% Em suma, a margem de segurança de 30\% confere resiliência ao projeto, protegendo-o contra incertezas inerentes ao desenvolvimento e operação de sistemas embarcados em um ambiente dinâmico. 

\subsection{Cálculo da Capacidade da Bateria}

O cenário de maior consumo para o dimensionamento da bateria é o que contempla todos os três lançamentos. Considerando que a energia total do sistema sem perdas para as três configurações é de aproximadamente 0,350 Wh, precisamos levar em conta a eficiência do circuito de conversão de energia. Devido às possíveis variações de tensão durante a operação e às perdas inerentes ao processo de elevação de tensão, adotamos um fator de eficiência de 0,85 como elevador de tensão. Assim, a energia requerida é calculada pela razão entre a energia total dos três lançamentos e esta eficiência.

\begin{equation}
E_{\text{requerida}}
\;=\;
\frac{E_{\text{total}}}{\eta_{\text{elevador}}}
\;=\;
\frac{0,350~\mathrm{Wh}}{0,85}
\;\approx\;
0,412~\mathrm{Wh}
\end{equation}

Aplicando a energia requerida pelo sistema já ajustado pela eficiência com a margem de segurança, chegamos ao valor real que a bateria precisa para que o projeto tenha sucesso na parte energética.

\begin{equation}
E_{\text{bateria}}
\;=\;
E_{\text{requerida}}
\times
\bigl(1 + \text{Margem de Segurança}\bigr)
\end{equation}

\begin{equation}
E_{\text{bateria}}
\;=\;
0,412~\mathrm{Wh}
\times
\bigl(1 + 0{,}30\bigr)
\;=\;
0,412~\mathrm{Wh}
\times
1{,}30
\;\approx\;
0,536~\mathrm{Wh}
\end{equation}

\subsubsection{Fonte de Alimentação para a Base de Lançamento}

% Fonte de Alimentação para a Base de Lançamento: Power Bank 

Para a Base de Lançamento, utilizaremos um Power Bank de 3300 mAh (5V, 2A) como fonte de energia. Esta escolha é prática e conveniente, pois power banks são dispositivos amplamente disponíveis, recarregáveis e projetados para fornecer energia de forma estável (geralmente 5V via USB, que será convertida para 3.3V pelo regulador interno do ESP32 da base, ou por um regulador externo se o power bank for de 3.3V) 

\begin{equation}
E_{\text{power bank}}
= 3.3~\mathrm{Ah} \times 5~\mathrm{V}
= 16.5~\mathrm{Wh}
\end{equation}

Mesmo com a margem de segurança de 30\% e a eficiência de conversão interna do power bank e dos reguladores dos componentes (se aplicáveis), a capacidade de um power bank de 3300 mAh é muito mais do que suficiente para o consumo da base (aproximadamente 0,042 Wh por lançamento), garantindo longos períodos de operação e múltiplos lançamentos sem a necessidade de recarga frequente. 

% Fonte de Alimentação para o Foguete: Bateria de Lítio (LiPo) 

Para o subsistema embarcado no foguete, a leveza e a alta densidade energética são cruciais. Portanto, utilizaremos uma bateria de Polímero de Lítio (LiPo). 

O cálculo da energia requerida da bateria do foguete, considerando a eficiência de 85\% do circuito elevador de tensão e a margem de segurança de 30\%, é baseado no consumo de 268,54 J por lançamento. Para o cenário de três lançamentos, a energia total do foguete (sem considerar perdas e margem) seria de
\begin{equation}
E_{\text{total}} = 3 \times 0{,}075~\mathrm{Wh} = 0{,}225~\mathrm{Wh}
\end{equation}

Aplicando a eficiência do elevador de tensão e a margem de segurança especificamente para o consumo do foguete. A energia total de 3 lançamentos é 0,225 Wh. A energia requerida devido à eficiência do circuito de elevação de tensão, que considera as possíveis variações de tensão durante a operação, é calculada como:

\begin{equation}
E_{\mathrm{bateria,foguete}}
=
\frac{0{,}225~\mathrm{Wh}}{0{,}85}
\approx
0{,}265~\mathrm{Wh}
\end{equation}

Já a energia requerida com margem de segurança:

\begin{equation}
E_{\mathrm{bateria,foguete}}
=
0{,}265~\mathrm{Wh}
\times
\bigl(1 + 0{,}30\bigr)
=
0{,}265~\mathrm{Wh}
\times
1{,}30
\approx
0{,}345~\mathrm{Wh}
\end{equation}

Com a energia requerida de aproximadamente 0,345 Wh, e considerando a utilização de um elevador de tensão genérico (que permite o uso de uma LiPo 1S de 3,7 V nominal), calculamos a capacidade necessária para a bateria do foguete.

Para o circuito de elevação de tensão, adotamos um fator de eficiência de 0,85, valor que leva em conta as possíveis variações de tensão durante a operação e as perdas inerentes ao processo de conversão. Assim, para calcular a capacidade em mAh necessária para uma LiPo de 3,7V:

\begin{equation}
\mathrm{Capacidade\,(mAh)}
=
\frac{E_{\mathrm{bateria,foguete}} \times 1000}{V_{\mathrm{nominal}}}
\end{equation}

\begin{equation}
\mathrm{Capacidade\,(mAh)}
=
\frac{0{,}345~\mathrm{Wh} \times 1000}{3{,}7~\mathrm{V}}
\approx
93~\mathrm{mAh}
\end{equation}

Portanto, para o sistema eletrônico embarcado dentro do foguete, utilizaremos uma bateria LiPo de 3,7 V (1S) com capacidade mínima de 100 mAh (ou 150 mAh para maior folga).

\subsection{Conclusão sobre as Fontes de Alimentação}

A base de lançamento é alimentada por um power bank de 3300 mAh (5V). Já que power banks são dispositivos amplamente disponíveis, recarregáveis e projetados assim para fornecer energia de forma estável. E com uma capacidade energética de 16.5 Wh, o nosso power bank oferece uma reserva de energia que excede significativamente o consumo estimado da base (que dá aproximadamente 0,042 Wh por lançamento). Isso então garante vários lançamentos e longos períodos de teste sem a necessidade de recarga frequente para a plataforma de lançamento. 